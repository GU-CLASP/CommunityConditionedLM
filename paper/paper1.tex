% This must be in the first 5 lines to tell arXiv to use pdfLaTeX, which is strongly recommended.
%\pdfoutput=1
% In particular, the hyperref package requires pdfLaTeX in order to break URLs across lines.

\documentclass[11pt]{article}

% Remove the "review" option to generate the final version.
\usepackage[review]{emnlp2021}

% Standard package includes
\usepackage{times}
\usepackage{latexsym}

% For proper rendering and hyphenation of words containing Latin characters (including in bib files)
\usepackage[T1]{fontenc}
% For Vietnamese characters
% \usepackage[T5]{fontenc}
% See https://www.latex-project.org/help/documentation/encguide.pdf for other character sets

% This assumes your files are encoded as UTF8
\usepackage[utf8]{inputenc}

% This is not strictly necessary, and may be commented out,
% but it will improve the layout of the manuscript,
% and will typically save some space.
\usepackage{microtype}

\usepackage{amsmath}
\usepackage{amssymb}
\usepackage{siunitx}
\usepackage{todonotes}
\usepackage{enumitem}
\usepackage{xcolor}
\usepackage{cleveref}
\usepackage{booktabs}
\usepackage{array}
\usepackage{makecell}
\usepackage{multirow}
\usepackage{longtable}
\usepackage{capt-of}
\usepackage[multiple]{footmisc}

\usepackage{pgfplots}
\usetikzlibrary{arrows,patterns}
\usepgfplotslibrary{groupplots}
\usepgfplotslibrary{statistics}
\pgfplotsset{compat=1.16}
\pgfplotsset{every tick label/.append style={font=\tiny}}
\usepackage{pgfplotstable}

\renewcommand{\UrlFont}{\ttfamily\small}

\newcommand\jp[1]{\todo[backgroundcolor=blue!10]{JP: #1}}
\newcommand\bn[1]{\todo[backgroundcolor=green!10]{BN: #1}}

\newcommand\Ppl{\mathsf{Ppl}}
\newcommand\IG{\mathsf{IG}}
\newcommand\Ind{\mathsf{Ind}}
\newcommand\LM{\mathrm{LM}}
\newcommand\CCLM{\mathrm{CCLM}}
\DeclareMathOperator*{\avg}{average}
\newcommand\BibTeX{B\textsc{ib}\TeX}
\newcommand\subreddit[1]{\texttt{/r/#1}}
\newlist{hypotheses}{enumerate}{3}
\setlist[hypotheses,1]{parsep=0pt,itemsep=1pt,font=\bfseries,label=(H\arabic*)}


\usepackage{newunicodechar}
\newunicodechar{ }{{\,}}
\newunicodechar{¬}{\ensuremath{\neg}} % 
\newunicodechar{²}{^2}
\newunicodechar{³}{^3}
\newunicodechar{·}{\ensuremath{\cdot}}
\newunicodechar{¹}{^1}
\newunicodechar{×}{\ensuremath{\times}} % 
\newunicodechar{÷}{\ensuremath{\div}} % 
\newunicodechar{ʳ}{^r}
\newunicodechar{ˡ}{^l}
\newunicodechar{̷}{\not} % 
\newunicodechar{Γ}{\ensuremath{\Gamma}}   %
\newunicodechar{Δ}{\ensuremath{\Delta}} % 
\newunicodechar{Η}{\ensuremath{\textrm{H}}} % 
\newunicodechar{Θ}{\ensuremath{\Theta}} % 
\newunicodechar{Λ}{\ensuremath{\Lambda}} % 
\newunicodechar{Ξ}{\ensuremath{\Xi}} % 
\newunicodechar{Π}{\ensuremath{\Pi}}   %
\newunicodechar{Σ}{\ensuremath{\Sigma}} % 
\newunicodechar{Φ}{\ensuremath{\Phi}} % 
\newunicodechar{Ψ}{\ensuremath{\Psi}} % 
\newunicodechar{Ω}{\ensuremath{\Omega}} % 
\newunicodechar{α}{\ensuremath{\mathnormal{\alpha}}}
\newunicodechar{α}{\ensuremath{\mathnormal{\alpha}}} % 
\newunicodechar{β}{\ensuremath{\beta}} % 
\newunicodechar{γ}{\ensuremath{\mathnormal{\gamma}}} % 
\newunicodechar{δ}{\ensuremath{\mathnormal{\delta}}} % 
\newunicodechar{ε}{\ensuremath{\mathnormal{\varepsilon}}} % 
\newunicodechar{ζ}{\ensuremath{\mathnormal{\zeta}}} % 
\newunicodechar{η}{\ensuremath{\mathnormal{\eta}}} % 
\newunicodechar{θ}{\ensuremath{\mathnormal{\theta}}} % 
\newunicodechar{ι}{\ensuremath{\mathnormal{\iota}}} % 
\newunicodechar{κ}{\ensuremath{\mathnormal{\kappa}}} % 
\newunicodechar{λ}{\ensuremath{\mathnormal{\lambda}}} % 
\newunicodechar{μ}{\ensuremath{\mathnormal{\mu}}} % 
\newunicodechar{ν}{\ensuremath{\mathnormal{\mu}}} % 
\newunicodechar{ξ}{\ensuremath{\mathnormal{\xi}}} % 
\newunicodechar{π}{\ensuremath{\mathnormal{\pi}}}
\newunicodechar{π}{\ensuremath{\mathnormal{\pi}}} % 
\newunicodechar{ρ}{\ensuremath{\mathnormal{\rho}}} % 
\newunicodechar{σ}{\ensuremath{\mathnormal{\sigma}}} % 
\newunicodechar{τ}{\ensuremath{\mathnormal{\tau}}} % 
\newunicodechar{φ}{\ensuremath{\mathnormal{\varphi}}} % 
\newunicodechar{χ}{\ensuremath{\mathnormal{\chi}}} % 
\newunicodechar{ψ}{\ensuremath{\mathnormal{\psi}}} % 
\newunicodechar{ω}{\ensuremath{\mathnormal{\omega}}} %  
\newunicodechar{ϕ}{\ensuremath{\mathnormal{\phi}}} % 
\newunicodechar{ϵ}{\ensuremath{\mathnormal{\epsilon}}} % 
\newunicodechar{ᵏ}{^k}
\newunicodechar{ᵢ}{_i} % 
\newunicodechar{ }{{\,}}
\newunicodechar{′}{\ensuremath{^\prime}}  % 
\newunicodechar{″}{\ensuremath{^\second}} % 
\newunicodechar{‴}{\ensuremath{^\third}}  % 
\newunicodechar{ⁱ}{^i}
\newunicodechar{⁵}{\ensuremath{^5}}
\newunicodechar{⁺}{\ensuremath{^+}} %% 
\newunicodechar{⁺}{^+}
\newunicodechar{⁻}{\ensuremath{^-}} %% 
\newunicodechar{ⁿ}{^n}
\newunicodechar{₀}{\ensuremath{_0}} % 
\newunicodechar{₁}{\ensuremath{_1}}
\newunicodechar{₁}{_1}
\newunicodechar{₂}{\ensuremath{_2}}
\newunicodechar{₂}{_2}
\newunicodechar{₃}{\ensuremath{_3}}
\newunicodechar{₊}{\ensuremath{_+}} %% 
\newunicodechar{₋}{\ensuremath{_-}} %% 
\newunicodechar{ℂ}{\ensuremath{\mathbb{C}}} % 
\newunicodechar{ℕ}{\mathbb{N}} % 
\newunicodechar{ℚ}{\ensuremath{\mathbb{Q}}}
\newunicodechar{ℝ}{\ensuremath{\mathbb{R}}} % 
\newunicodechar{ℤ}{\ensuremath{\mathbb{Z}}} % 
\newunicodechar{ℳ}{\mathscr{M}}
\newunicodechar{⅋}{\ensuremath{\parr}} % 
\newunicodechar{←}{\ensuremath{\leftarrow}} % 
\newunicodechar{↑}{\ensuremath{\uparrow}} % 
\newunicodechar{→}{\ensuremath{\rightarrow}} %
\newunicodechar{↔}{\ensuremath{\leftrightarrow}} % 
\newunicodechar{↖}{\nwarrow} %
\newunicodechar{↗}{\nearrow} %
\newunicodechar{↝}{\ensuremath{\leadsto}}
\newunicodechar{↦}{\ensuremath{\mapsto}}
\newunicodechar{⇆}{\ensuremath{\leftrightarrows}} % 
\newunicodechar{⇐}{\ensuremath{\Leftarrow}} % 
\newunicodechar{⇒}{\ensuremath{\Rightarrow}} % 
\newunicodechar{⇔}{\ensuremath{\Leftrightarrow}} % 
\newunicodechar{∀}{\ensuremath{\forall}}   %
\newunicodechar{∂}{\ensuremath{\partial}}
\newunicodechar{∃}{\ensuremath{\exists}} % 
\newunicodechar{∅}{\ensuremath{\varnothing}} % 
\newunicodechar{∈}{\ensuremath{\in}}
\newunicodechar{∉}{\ensuremath{\not\in}} % 
\newunicodechar{∋}{\ensuremath{\ni}}  % 
\newunicodechar{∎}{\ensuremath{\qed}} % end of proof. % or: {\blacksquare} 
\newunicodechar{∏}{\prod}
\newunicodechar{∑}{\sum}
\newunicodechar{∗}{\ensuremath{\ast}} % 
\newunicodechar{∘}{\ensuremath{\circ}} % 
\newunicodechar{∙}{\ensuremath{\bullet}} % 
\newunicodechar{∙}{\ensuremath{\cdot}}
\newunicodechar{∞}{\ensuremath{\infty}} % 
\newunicodechar{∣}{\ensuremath{\mid}} % 
\newunicodechar{∧}{\ensuremath{\wedge}}% 
\newunicodechar{∨}{\ensuremath{\vee}}% 
\newunicodechar{∩}{\ensuremath{\cap}} % 
\newunicodechar{∪}{\ensuremath{\cup}} % 
\newunicodechar{∫}{\int}
\newunicodechar{∷}{::} % proportion
\newunicodechar{∼}{\ensuremath{\sim}} % 
\newunicodechar{≃}{\ensuremath{\simeq}} % 
\newunicodechar{≅}{\ensuremath{\cong}} % 
\newunicodechar{≈}{\ensuremath{\approx}} % 
\newunicodechar{≜}{\ensuremath{\stackrel{\scriptscriptstyle {\triangle}}{=}}} 
\newunicodechar{≜}{\triangleq}
\newunicodechar{≝}{\ensuremath{\stackrel{\scriptscriptstyle {\text{def}}}{=}}}
\newunicodechar{≟}{\ensuremath{\stackrel {_\text{\textbf{?}}}{\text{\textbf{=}}\negthickspace\negthickspace\text{\textbf{=}}}}} % or {\ensuremath{\stackrel{\scriptscriptstyle ?}{=}}}
\newunicodechar{≠}{\ensuremath{\neq}}% 
\newunicodechar{≡}{\ensuremath{\equiv}}% 
\newunicodechar{≤}{\ensuremath{\le}} % 
\newunicodechar{≥}{\ensuremath{\ge}} % 
\newunicodechar{⊂}{\ensuremath{\subset}} % 
\newunicodechar{⊃}{\ensuremath{\supset}} %  
\newunicodechar{⊆}{\ensuremath{\subseteq}} %  
\newunicodechar{⊇}{\ensuremath{\supseteq}} % 
\newunicodechar{⊎}{\ensuremath{\uplus}} %
\newunicodechar{⊑}{\ensuremath{\sqsubseteq}} % 
\newunicodechar{⊒}{\ensuremath{\sqsupseteq}} % 
\newunicodechar{⊓}{\ensuremath{\sqcap}} % 
\newunicodechar{⊔}{\ensuremath{\sqcup}} % 
\newunicodechar{⊕}{\ensuremath{\oplus}} % 
\newunicodechar{⊗}{\ensuremath{\otimes}} % 
\newunicodechar{⊢}{\ensuremath{\vdash}} % 
\newunicodechar{⊥}{\ensuremath{\bot}} % bottom
\newunicodechar{⊧}{\models} % 
\newunicodechar{⊨}{\models} % 
\newunicodechar{⊩}{\Vdash}
\newunicodechar{⊸}{\ensuremath{\multimap}} % 
\newunicodechar{⋁}{\ensuremath{\bigvee}}
\newunicodechar{⋃}{\ensuremath{\bigcup}} % 
\newunicodechar{⋄}{\ensuremath{\diamond}} % 
\newunicodechar{⋅}{\ensuremath{\cdot}}
\newunicodechar{⋆}{\ensuremath{\star}} % 
\newunicodechar{⋮}{\ensuremath{\vdots}} % 
\newunicodechar{⋯}{\ensuremath{\cdots}} % 
\newunicodechar{─}{---}
\newunicodechar{■}{\ensuremath{\blacksquare}} % black square
\newunicodechar{□}{\ensuremath{\square}} % 
\newunicodechar{▹}{\ensuremath{\rhd}} % 
\newunicodechar{◃}{\triangleleft{}}
\newunicodechar{◅}{\ensuremath{\triangleleft{}}}
\newunicodechar{◇}{\ensuremath{\diamond}} % 
\newunicodechar{◽}{\ensuremath{\square}}
\newunicodechar{★}{\ensuremath{\star}}   %
\newunicodechar{♭}{\ensuremath{\flat}} % 
\newunicodechar{♯}{\ensuremath{\sharp}} % 
\newunicodechar{✓}{\ensuremath{\checkmark}} % 
\newunicodechar{⟂}{\ensuremath{^\bot}} % PERPENDICULAR 
\newunicodechar{⟦}{\ensuremath{\llbracket}}
\newunicodechar{⟧}{\ensuremath{\rrbracket}}
\newunicodechar{⟨}{\ensuremath{\langle}} % 
\newunicodechar{⟩}{\ensuremath{\rangle}} % 
\newunicodechar{⟶}{{\longrightarrow}}
\newunicodechar{⟷} {\ensuremath{\leftrightarrow}}
\newunicodechar{⟹}{\ensuremath{\Longrightarrow}} % 
\newunicodechar{𝒟}{\ensuremath{\mathcal{D}}} % 
\newunicodechar{𝒢}{\ensuremath{\mathcal{G}}}
\newunicodechar{𝒦}{\ensuremath{\mathcal{K}}} % 
\newunicodechar{𝒫}{\ensuremath{\mathcal{P}}}
\newunicodechar{𝔸}{\ensuremath{\mathds{A}}} % 
\newunicodechar{𝔹}{\ensuremath{\mathds{B}}} % 
\newunicodechar{𝟙}{\ensuremath{\mathds{1}}} % 
\newunicodechar{ℒ}{\ensuremath{\mathscr{L}}}
\newunicodechar{⊤}{\ensuremath{\top}}
\newunicodechar{ }{\quad}
\newunicodechar{⊛}{\ensuremath{\circledast}}



\title{Community-Conditioned Language Models}

% Author information can be set in various styles:
% For several authors from the same institution:
% \author{Author 1 \and ... \and Author n \\
%         Address line \\ ... \\ Address line}
% if the names do not fit well on one line use
%         Author 1 \\ {\bf Author 2} \\ ... \\ {\bf Author n} \\
% For authors from different institutions:
% \author{Author 1 \\ Address line \\  ... \\ Address line
%         \And  ... \And
%         Author n \\ Address line \\ ... \\ Address line}
% To start a seperate ``row'' of authors use \AND, as in
% \author{Author 1 \\ Address line \\  ... \\ Address line
%         \AND
%         Author 2 \\ Address line \\ ... \\ Address line \And
%         Author 3 \\ Address line \\ ... \\ Address line}

\author{First Author \\
  Affiliation / Address line 1 \\
  Affiliation / Address line 2 \\
  Affiliation / Address line 3 \\
  \texttt{email@domain} \\\And
  Second Author \\
  Affiliation / Address line 1 \\
  Affiliation / Address line 2 \\
  Affiliation / Address line 3 \\
  \texttt{email@domain} \\}

\date{}


\begin{document}
\maketitle
\begin{abstract}
  Community-level linguistic variation is a core concept in sociolinguistics,
  but more tools are needed to study variation from a computational perspective.
  In this paper, we use conditional language models to investigate linguistic variation
  across 510 online communities.
  We investigate the community representations learned by the conditional language model
  and find that they are highly correlated with a social network-based 
  representation that doesn't use any linguistic input.
\end{abstract}


% Introduction

% Community Conditioned Language Models:
%   Data sets
%   Architecture (with emphasis on the embeddings)
%   Training
%   Results (loss and loss per community)

% Latent bayesian community classifier
%    Formula (no training!)
%    Result 1: Confusion matrix
%    Transition: too much data; can we extract a more useful metric?
%    Indiscernibility: Formula, and table

% Correlation between linguistic embedding with social embeddings
%    Kumar2018 Data set
%    Explain the problem (idea: comparing cosine similarities --> too much data, again)
%    Sketch "procrustes" problem, explain what it yields. (Also what kind of similarity one gets with random input)
%    Results:
%      - Correlation between Kumar2018 and best LSTM
%      - Correlation between Kumar2018 and best Transformer 
%      - Correlation between Transformer and LSTM
%      - Concatenations ???



\section{Introduction}
Linguistic communication requires that speakers share
knowledge of certain linguistic conventions, such as syntactic
structure, word meanings, and accepted patterns of interaction.
Speakers assume that these conventions are \emph{common ground} among
their interlocutors, based on joint membership in a community
\cite{Stalnaker2002, Clark1996}.  Such a community may be of any size,
from the very small, like members of a particular friend group, to the
very large, like speakers of English.  Because it is the source of
linguistic convention, the \emph{speech community} is an important
concept in the study of linguistic variation \citep{Gumperz1972}.

% Not only is linguistic variation an object of inquiry in its own
% right, but it is also important to consider it in the development of
% robust, equitable language technology.  Previous work has shown that
% by taking demographic information into account, models can achieve
% better performance on various standard NLP tasks
% \citep{Hovy2015,Yang2017}. \jp{Move this to related work so we can get to the point quicker.}
Most of the previous computational work on linguistic variation 
has considered variation at the level of macro-social categories, such as gender
\citep{Burger2011,Ciot2013,Bamman2014}, age \citep{Nguyen2013}, and
geographic location \citep{Eisenstein2010,Bamman2014a}.
Some recent exceptions are \citet{DelTredici2017} who use a modified
skip-gram model to show that lexical semantic variation occurs across 
different communities organised around the same topic and 
\citet{Lucy2021}, who use contextualised word representations to
investigate community-level variation in word sense distributions.

In the present work, we investigate linguistic variation
across different online communities. In contrast to previous work, we
investigate variation across speech communities (rather than
macro-social categories).


%\jp{
  %Some examples (perhaps include if there is space):
%  
  %(CFB)
  %It was against SMU in the 47-48 Cotton bowl. They had cancelled a
  %game the year before against Miami for the same reason. There's
  %debate on if that's really the origin for the phrase but the story
  %itself is true.
%
  %(Videos)
  %Reminds me of Y2K computer bug. A lot of work went into
  %pre-emptively fixing it, and a lot of bugs were found and fixed,
  %some of which would have been catastrophic. Unfortunately, because
  %the work was successful, a lot of people now describe it as a waste
  %of time/money and a big deal made out of nothing.  }
%
% Indeed, it is folklore that certain communities tend to exhibit more predictable
% speech patterns than others.  Here we want to quantify this
% phenomenon.

For this purpose, we define (\cref{sec:cclm}) 
various Community Conditioned Language Models 
(CCLM for short).
% , and model the degree to which a
% given message $m$ is acceptable in a given community $c$, by estimating the probability
% \(P(m \mid c)\).

We want to find out how, and how much, CCLMs are attuned to
In practice, we test whether the linguistic
models are correlated with a social network model of communities (based only on
co-occurrence of users) (\cref{sec:embedding-analysis})

% To sum up, we test the following hypotheses, which we review in our
% conclusion (\cref{sec:discussion-conclusion}):
% \begin{hypotheses}
% \item \label{hyp:varying-complexity} Different communities have different levels of linguistic
%   complexity
% \item \label{hyp:layer-effect} The layer at which the community embedding is taken into account
%   in a CCLM influences its perplexity.
% \item \label{hyp:LMCC-works} CCLMs can be used to recognise communities based on
%   latent Bayesian classification.
% % \item \label{hyp:LMCC-works} The layer at which the community embedding is taken into account
% %   influences CCLM and its associated latent classifier.
%   % This is more of a justification for studying the latent classifier than a real "user-facing" hypothesis.
% \item \label{hyp:extra-linguistic-correlation} The CCLM representation of communities are correlated with
%   co-occurrence of users between communities.
% % \item Community stability is correlated with language complexity
% \item \label{hyp:rich-harder-to-identify} Linguistically poor communities are easier to identify than
%   linguistically rich communities.
% \end{hypotheses}

% We proceed as follows: 
% First, we define the CCLM models used in the paper, as well as baseline un-conditioned madels.
% We describe how the models were trained and report on overall perplexity results, 
% as well as comparing the perplexity of the CCLM models to the baseline at the community level
% .
% Next, we define latent Bayesian community classifiers based on the CCLM. We define a measure
% of \emph{indiscernability} for communities and report on correlations between indiscernability and perplexity .
% Then, we focus on the community embeddings learend by the CCLM,
% comparing them to a social network-based embedding .
% Finally, we give an overview of related work and discuss our findings in that context .


\section{Community-conditioned language models}\label{sec:cclm}

We experiment with two kinds of model architecture: a simple
unidirectional LSTM \citep{Hochreiter1997} and a Transformer
\citep{Vaswani2017}.  In either case, the model is organised as a
standard $n$-layer neural sequence encoder (we set $n=3$ for all
models). As is convention, a sequence of word tokens is first converted to 
vectors using a trainable embedding layer over a pre-determined vocabulary.
At the other end, word tokens are predicted using a softmax projection 
layer. What we have described so far does not take community into account
and as such we call them \emph{uncodnitioned models}, but the same
encoder architecture also forms the core of our conditioned models.

To condition the model , we add a \emph{community embedding} parameter.
This parameter is concatenated with the hidden layer of the
sequence encoder, at some layer $l_c \leq n$, and passed through a
linear layer which projects the resulting vector back to the original
hidden layer size.  For $l_c = n$, the output of this linear layer is
passed directly to the softmax function, just as the final hidden
layer of the sequence encoder is in other models.  For $l_c=0$, the
community embedding is concatenated with the token embedding.  For
this reason, we set the hidden size of the sequence encoder and the
size of the token embedding to be equal for all models.

%TODO: Diagram of model?

\subsection{Data sets}

We investigate linguistic variation across various communities 
from the social media website Reddit.\footnote{Comments were obtained
  from the archive at \url{https://pushshift.io/}.
  \cite{Baumgartner2020}.}
%
Reddit is divided into forums called \textit{subreddits}, 
which are typically organised around a topic of interest. 
Users create \textit{posts}, which consist of a link, image, 
or text, along with a \emph{comment} section. 
Comments are threaded: a comment can be made directly on a post,
or appear as a reply to another comment.
%
Hereafter we refer to such comments as ``messages'', matching our
convention in mathematical formulas: the letter $c$ stands for a
community, and $m$ stands for a message.

Our dataset includes messages from \num{510} subreddits, 
the set of all subreddits 
with at least \num{5000} messages per month for each month
of the year 2015.
Most but not all of these communities (with about 10 exceptions) 
are mostly English-language.
Each community corpus consist of \num{42000} randomly selected messages from the year 2015.
For each community, we reserve \num{1000} messages each for development and testing,
leaving a total of \num{20.4}M messages for training.

Messages were preprocessed as follows. 
We excluded the content of block quotes, code blocks, and tables.
We removed any markup (formatting) commands from the remaining text, 
extracting only rendered text.
We tokenized the messages using the default English model for the SpaCy tokenizer 
Version 2.2.3 \citep{Honnibal2017}.


\subsection{Training scheme}

We use a vocabulary size of \num{40000} tokens, including a special
out-of-vocabulary token.  The vocabulary consists of the most frequent
tokens across all communities.

For the LSTM architecture, we trained the models on a simple left-to-right language
modeling task with cross-entropy loss.  Because
the transformer operates on all tokens in the sequence at once, the
language modeling was achieved by masking and incrementally un-masking
input tokens in a left-to-right fashion (so we have an
auto-regressive model).  We used the AdamW
\citep{Loshchilov2019} optimisation algorithm, with an initial
learning rate of \num{0.001}, with no extra control on the decay
of learning rate.
%
We used a batch size of \num{256} and a maximum sequence length of
\num{64} tokens, truncating longer messages (16.8\% of messages were
longer than \num{64} tokens).  During training, a dropout rate of
$0.1$ was applied between encoder layers and after each linear layer.

All experiments use models with \num{3} sequence encoder layers,
each with hidden (and token embedding) size of \num{256}. 
The transformer models had \num{8} attention heads per layer.\footnote{
  This number of attention heads was chosen in part to give the LSTM and transformer
  models a comparable number of parameters 
  (\num{22171203} and \num{21779523}, respectively).}
The conditional models were given a community embedding with \num{16} dimensions. 
We experimented with every possible value for $l_c$ in a three-layer model ($l_c\in\{0,1,2,3\}$).

We trained the models until the validation loss stopped decreasing for
two epochs in a row, and used the weights from the epoch with the
smallest validation loss for testing.  In general, transformer models
were trained for about half as long as the LSTM models (\cref{tab:model-results}, 
test epoch).

\subsection{Results and analysis}

In this section, we report the performance of the conditioned and un-conditioned
models on the test set.
First, we define two performance metrics, perplexity and information gain.
In the following, we use $M$ to refer to messages in the
combined test set, and $M_j$ for the partition of the test set originating from 
community $c_j$.

\paragraph{Perplexity}

For a given model let $H(m)$ be the model's cross-entropy loss,
averaged over tokens in $m$.
We define the perplexity on a set of messages, $M$,
to be the exponential of the model's average cross-entropy loss:
\[\Ppl_M = e^{\avg_{m\in M} H(m)}\]

\paragraph{CCLM Information Gain}

We also consider the average information gain per token of the CCLM over its baseline
un-conditioned counterpart, with the same sequence encoder architecture.
For a given message, information gain is defined as the difference
between the cross-entropy of the unconditioned model and the conditioned model:
\[H_{\mathrm{LM}}(m) - H_{\mathrm{CCLM}}(m)\]
For a set of messages, $M$, we consider the average information gain
in exponential space (as a ratio of perplexities):
\[\IG_M = \frac{e^{\avg_{m \in M}(H_{\mathrm{LM}}(m))}}{e^{\avg_{m \in M}(H_{\mathrm{CCLM}}(m))}}\]
\[\IG_M = {e^{\avg_{m \in M}(H_{\mathrm{LM}}(m) - H_{\mathrm{CCLM}}(m))}}\]
%
Unsurprisingly, the conditioned models mostly have lower perplexity 
than their respective unconditioned baseline models, 
meaning information gain above one (\cref{tab:model-results}).
Overall, the LSTM models perform better than the transformers, 
although the best transformer has higher information gain than the best LSTM.

The effect of $l_c$, the depth of the community embedding,
is also different across architectures.
For the LSTM encoder, 
the best model concatenates the community embedding after the first encoder layer ($l_c=1$),
but all of the conditioned models perform similarly well.
For the transformer, the best model incorporates the community information
first, concatenating it directly to the word vectors ($l_c=0$).
It performs similarly to the model that only integrates the community information
after all all the transformer layers ($l_c=3$),
but the two middle-layer models actually perform worse than the unconditioned model
(with $\IG_M < 1$).

\begin{table}
  \small
  \centering
  \begin{tabular}{llrrr}
  \toprule
                                 &       & \makecell{test \\ epoch} & $\Ppl_M$ & $\IG_M$     \\
                                 & $l_c$ &         &                 &                   \\
  \midrule
  \multirow{5}{*}{LSTM}         & - &          12 &  68.74          &       -           \\
                                & 0 &          13 &  66.16          &    1.039          \\
                                & 1 &           7 &  \textbf{66.01} &    \textbf{1.041} \\
                                & 2 &           4 &  66.19          &    1.039          \\
                                & 3 &           4 &  66.35          &    1.036          \\
  \midrule
  \multirow{5}{*}{Transformer}  & - &           4 &  79.13          &        -          \\
                                & 0 &           4 &  \textbf{75.66} &    \textbf{1.046} \\
                                & 1 &           4 &  82.12          &    0.964          \\
                                & 2 &           7 &  83.53          &    0.947          \\
                                & 3 &           3 &  75.90          &    1.043          \\
  \bottomrule
  \end{tabular}
  \caption{
    Performance of baseline (first row for each encoder architecture)
    and CCLM models.  The scope of perplexity and information gain
    ($M$) is the entire test set, i.e. $5000×510$ messages; \num{5000}
    for each community.}
  \label{tab:model-results}
\end{table}

We also consider performance stratified by community; that is,
$\Ppl_{M_j}$ and $\IG_{M_j}$, 
where $M_j$ is the set of messages originating from community $c_j$ 
(\cref{fig:comm-stratified-box}).
We observe a lot of variation in baseline perplexity
across communities, with $\Ppl_{M_j}$ ranging from \num{3.67} to
\num{93.58} for the best conditional LSTM model 
(see supplementary materials for perplexity by community). %TODO keep? 
While the absolute performance ($\Ppl_{M_j}$) of the LSTM models is better,
the best transformer models have somewhat higher information gain
than their LSTM counterparts.


\newcommand{\modelboxplot}[3]{
  \addplot+[
    boxplot={draw position=#3, box extend=0.3}, 
    draw=#2, mark=*, mark options={fill=#2, scale=0.5}, solid, fill=#2!10,
    area legend] 
    table [y=#1] {floats/comm.csv};
}
\begin{figure}[t]
\begin{tikzpicture}
  \begin{axis}[name=ig,
    height=7cm,width=8cm,
    boxplot/draw direction=x, axis x line*=bottom, axis y line*=left,y axis line style={opacity=0},
    xmax=1.6,
    ylabel=$l_c$, xlabel=$\IG_{M_j}$,
    legend style={font=\tiny,at={(1,1)},anchor=north east}, legend columns=2
    ]
    \addplot[thin, red, samples=2] coordinates {(1,-0.5)(1,3.5)};
    \foreach \lc in {0, ..., 3}{
      \modelboxplot{lstm-3-\lc-ig}{violet}{\lc+0.2}
      \modelboxplot{transformer-3-\lc-ig}{teal}{\lc-0.2}
    \legend{,LSTM, Transformer}
    }
  \end{axis}
  \begin{axis}[name=ppl,
    anchor=north west,at={($(ig.south west)-(0,1.1cm)$)},
    width=8cm, height=2.5cm, 
    boxplot/draw direction=x, axis x line*=bottom, hide y axis,
    xlabel=$\Ppl_{M_j}$,
    ]
    \modelboxplot{lstm-3-ppl}{violet}{0.2}
    \modelboxplot{transformer-3-ppl}{teal}{-.2}
  \end{axis}
\end{tikzpicture}
\caption{
  Average model performance by community.
  The boxes indicate upper and lower quartiles, while the whiskers are placed at the upper and lower maximum, with communities more than $1.5\times IQR$ (inter-quartile range) above the upper quartile considered outliers (represented as dots). The three most extreme outliers are excluded from this view (see \cref{fig:ppl-info-gain}).
   %See also table 1 in the supplementary materials % not very useful for this actually since we sort by indisc and it's only one model anyway.
   %for a tabular version of these results with community names.
}
\label{fig:comm-stratified-box}
\end{figure}

The conditioned models also perform differently across different communities---%
even among the best models, some communities have $\IG_{M_j} < 1$,
meaning that the CCLM performs worse than the unconditioned baseline for 
messages from that community.
For other communities $\IG_{M_j}$ is much higher, meaning that the CCLM performs better  
(\cref{fig:comm-stratified-box}).\footnote{
  Some of the communities with consistently high $IG_{M_j}$ 
  across all models are primarily non-English, 
  but surprisingly, not the three most extreme
  outliers. There are \subreddit{counting}, \subreddit{friendsafari}, and \subreddit{Fireteams}, 
  the later two of which are places where people coordinate to play video games
  together. The coordination has highly structured and regular formats, which 
  are presumably conventional to these communities.}
%
%The variation in information gain across communities is not 
%related to perplexity, however (\cref{fig:ppl-info-gain}).

We observe empirically that across all the models we
tested, communities where conditioning has the least effect tend to be
more organised around more general interest topics, such as 
\subreddit{relationships} and \subreddit{advice}, where the subject
matter is relevant to a broad range of people.  Conditioning the model
on community has the most benefit for narrower special-interest
subreddits, such as those organised around a certain videogame, sports
team, or subculture.  This makes sense intuitively, since communities
with more niche topics would tend to have more specialised vocabulary
and other linguistic patterns.

\section{Comparison of CCLM community embeddings with a social network embedding}\label{sec:embedding-analysis}

In this section we investigate whether (and how much) the community
embeddings obtained from the CCLMs correlate with the community-member
relationship on Reddit.

To this end, we compare the CCLM-learned community
embedding with the community embedding created by \citet{Kumar2018},
who generated them using using a negative-sampling optimization
algorithm and the message-community co-occurrence matrix as ground
truth, using data which spans January 2014 to April 2017.
We refer the reader
to \citet{Kumar2018}
for details, but the important point is that no linguistic
information is used to create these embeddings: they only reflect the social
relationship between communities, via 
the user-community membership relation. In contrast, CCLM community embeddings depend in no
way on which user is the author of any given message: we only use the
contents of messages, not authorship data.
%

\subsection{Comparing embeddings: anecdotal evidence}
\label{sec:storytelling}
TODO: BILL

\subsection{Comparing embeddings: statistical evidence}

The anecdotal evidence suggests that for some communities, proximity
in linguistic patterns do match social proximity… and for others it
does not. Unfortunately, it seems hard, if not impossible, to draw a
general conclusion from this way. In this section, we propose a
systematic approach thanks to which we can quantify the correlation
between social proximity and linguistic proximity, and measure its
statistical significance.

Let us call \(L\) the matrix of (normed) linguistic embeddings and
\(S\) the matrix of (normed) social embeddings. Thus \(L_i\) and
\(S_i\) are the linguistic and social embeddings for community \(i\).
%
A straightforward, but ultimately flawed, generalization of the
methodology used to collect data in \cref{sec:storytelling} is to
consider each pair of communities \((i,j)\), and compute the
correlation between the cosine similarities of both embeddings. That
is, compute the Pearson correlation factor of the data set:

\[\C = {(x = L_i · L_j, y = S_i · S_j) \text {~for~} i,j ∈ [1,512]\}\]

One could then use a standard statistical test to verify to obtain a
p-value for the correlation [TODO: what do we get?].  The flaw of
this methodology is that the number of pairs grows with the square
of the number of communities. With 510 communities, we have 129795
pairs, and standard statistical tests on Pearson correlation will
assure that we have a statistically significant correlation in all
but the weakest correlation. A further flaw is that the distribution of data points
of \(C\) are \emph{not} independent. Far from it, the are generated
from 1/510 independent variables. We consider this last flaw fatal, and take the following
approach instead.


% Explain the problem (idea: comparing cosine similarities --> too much data, again)

Instead of comparing embedding pairs, we will compare embeddings
community by community.

At this stage, one might naively think that it would suffice to
calculate the distance between two embeddings index-wise, which is
equal to the Frobenius distance between \(L\) and \(S\):

\[||L-S||_F = \sum_i (L_i - S_i)\]

The problem with this approach is that even if several dimensions of
\(L\) and \(S\) are correlated, they will not coincide in the
\emph{representation}. That is, applying a simple rotation (orthogonal
transformation) on either matrix widely changes the \(||L-S||_F\)
correlation metric. Another way to consider the same problem is that,
applying such a rotation does not change the cosine similarities
considered in the previous section.

To make the metric independent of the representation (up to orthogonal
transformations), we compute the \emph{minimum} distance between
\(L_i\) and \(S_i\), for any orthogonal matrix \(\Omega\) applied to
\(L\):

\[d(L,S) = \mathsf{argmin}_\Omega ||ΩL-S||_F\]

Here, the orthogonal matrix \(Ω\) gives a map from linguistic embeddings to social embeddings.
The problem of computing \(d(L,S)\) is known as the orthogonal Procrustes problem~\citep{Gower2004}. The
solution is
\[d(L,S) = n - \mathsf {Tr} (Σ)\] where the matrix \(Σ\) is obtained by the singular value
decomposition (SVD) \(U^TΣV = LS^T\). Additionally the vectors of \(U\) and \(V\) give
the directions of correlation respectively of \(L\) and \(S\). That is, each
singular value \(\sigma_i\) (the elements of the diagonal matrix \(Σ\)), give a measure of how much
correlation there is between the directions \(U_i\) and \(V_i\).

As common when doing SVD, we can arrange \(U\), \(V\) and \(Σ\) such
that \(σ_i > σ_j\) iff \(i < j\). Doing so, the largest singular value
\(σ_0\) corresponds to the principal directions of correlation
(\(U_0\), \(V_0\)), \(σ_1\) to the second principal direction, etc.

The \(d(L,S)\) metric ranges from \(0\) (corresponding to perfect
correlation, obtained for example if \(L=S\)) to \(n\) (corresponding
to perfect orthogonality), where $n=510$ is the number of communities
considered.

Now, to test if \(d(L,S)\) corresponds to a significant correlation,
it suffices to check if its value is significantly larger than it
value for random linguistic embeddings \(L\). Even though this is the
distribution of \(d(L,S)\) for random embeddings is difficult to
compute analytically, one can instead evaluate it using a Monte Carlo
method.

Doing so, we observed that \(d(L,S)\) exhibits a mean of
\(μ_d=431.39\) and a (Bessel's-corrected) standard deviation
\(s_d=2.90\) in their distance from the social embedding,
\(S\). % luckily it seems that sample stdev is often written with an $s$ to differentiate from population stdev


Thus if the real \(d(L,S)\) is below the mean by several standard
deviations, we can safely assume that there is statistically
significant correlation between \(L\) and \(S\). A 4-sigma difference
has less than one percent chance of occurring randomly. In our case:
we observe a difference between 61 and 68 standard deviations
(\cref{tab:embedding-correlations}). This definitely indicates a
significant correlation.  Furthermore, by coming back to the
definition of \(d(L,S)\), we know that, on average, the cosine
similarity between \(ΩL\) and \(S\) is \(510 - 232.18 / 510\).  It
further means that if we obtain a linguistic embedding \(L_k\) for a
new community $k$, we can estimate its social embedding by \(ΩL_k\),
and the cosine similarity with its true social embedding \(S_k\) is
expected to be \((431.39 - 232.18) / 510\) --- accounting for
over-fitting effects by taking the average distance rather than the
maximum. % JP: TODO is the number from the table. this should probably be rewritten
In sum, it is clear that the CCLM embeddings predict some aspect the social-network embeddings--- but far from all of it.

To finish, we also give a sense of \emph{how} the correlation is manifested overall, by
analysis of the two principal components of correlation in the
linguistic embeddings, \(U_0\) and \(U_1\). To do so we plot the
projection of each embedding along their first two principle components
which, together with the corresponding singular values, gives an idea
of how much and in what way they differ (see~\cref{fig:pca-aligned}).

\newcommand{\PCAAligned}[0]{
  \begin{tikzpicture}
    \begin{groupplot}[%
      group style={
        group size=2 by 1,
        horizontal sep=0pt,
      },
      xmin=-1, xmax=1,
      ymin=-1, ymax=1,
      xtick={-0.5,0.5}, ytick={-0.5,0.5},
      axis x line=middle,
      axis y line=middle,
      axis line style={(-)},
      axis on top=true,
      width=0.55\textwidth,
      height=0.55\textwidth,
      scatter/classes={
        0={mark=x, blue},
        1={mark=+, orange},
        2={mark=o, green},
        3={mark=triangle, gray},
        4={mark=star, magenta},
        5={mark=diamond, red}
      },
      ]
      \nextgroupplot[
        %xlabel=$U_0$, ylabel=$U_1$,
        legend cell align=left,
        legend style={font=\tiny},
        legend entries = {
          offmychest confession relationships relationship\_advice SuicideWatch, 
          TrueReddit China europe nyc socialism,
          frugalmalefashion running MechanicAdvice personalfinance Flipping,
          darksouls skyrimmods dragonage Xcom MonsterHunter,
          amiugly Rateme ladybonersgw SchoolIdolFestival GoneWildPlus,
          facepalm CrappyDesign mildlyinfuriating cringepics oddlysatisfying,
        },
        %legend pos=outer south east
        legend to name=pcalegend
      ]
      \addplot[%
        scatter, only marks,
        mark options={thick, scale=0.75},
       scatter src=explicit symbolic,
        ] table [x={web-PCA0}, y={web-PCA1}, meta={web-cluster}] {floats/comm.csv};
      \nextgroupplot[
      %xlabel=$V_0$, ylabel=$V_0$
      ]
      \addplot[%
        scatter, only marks,
        mark options={thick, scale=0.75},
       scatter src=explicit symbolic,
        ] table [x={lstm-3-1-PCA0}, y={lstm-3-1-PCA1}, meta={web-cluster}] {floats/comm.csv};
    \end{groupplot}
  \end{tikzpicture} 
}

\begin{figure*}[t]
    \centering
    \PCAAligned 
  \caption{First two components of the aligned social (left) and linguistic (right) embeddings,
    where the lingusitic embedding is taken from the LTSM with $l_c=1$.
    Correlation between these directions is given by $\sigma_0 = 53.4$ and $\sigma_1 = 35.6$.
    Colors are assigned by k-means clustering of the social embedding. 
    This figure is reproduced in the supplementary materials with a legend that helps 
    characterise the clusters.
  }
  \label{fig:pca-aligned}
\end{figure*}



\begin{table}
\centering
\begin{tabular}{llcc}
\toprule
                        &   & LSTM     & Transformer \\
\midrule
  \multirow{4}{*}{$l_c$}  & 0 &         254.06  (61.21) &         239.41  (66.79)  \\
                          & 1 &         245.14  (64.29) & \textbf{232.18} (68.54)   \\
                          & 2 &         249.17  (62.90) &         233.47  (68.32)  \\
                          & 3 & \textbf{241.13} (65.67) &         237.74   (66.84) \\
\bottomrule
\end{tabular}
\caption{Distance between CCLM embeddings and the social network-based embedding
of \citet{Kumar2018}, as measured by $d(L,S)$. In parentheses is the number of standard deviations
from the mean distance of our random embedding samples.} \label{tab:embedding-correlations}
\end{table}

% (Another idea may be to correlate the cosine similarity between
% \((S_i,S_j)\) and \((L_i,L_j)\). But this would yield one number for
% each pair of communities \((510 × 509 / 2)\): too much data to deal with,
% and it is hard to get a sense of the meaning, let alone statistical significance of the
% resulting correlation coefficient.)\jp{Not sure if this discussion has its place here.}


% \subsection{Results}

%    Results:
%      - Correlation between Kumar2018 and best LSTM
%      - Correlation between Kumar2018 and best Transformer 
%      - Correlation between Transformer and LSTM
%      - Concatenations ???


%\begin{figure*}[t]
%\begin{tikzpicture}
  %\begin{groupplot}[
    %group style={group size=2 by 1},
    %yticklabels={,,}, xticklabels={,,},
    %legend style={anchor=north, legend columns=1, font=\sffamily\tiny},
  %]
    %\nextgroupplot
    %\commscatter{lstm-3-1-pca1}{lstm-3-1-pca2}
    %\nextgroupplot
    %\commscatter{transformer-3-3_pca1}{transformer-3-3_pca2}
    %\legend{General interest,Videogames,Technology,Sports,Female-focused,Other}
  %\end{groupplot}
%\end{tikzpicture}
%\caption{Plot of embeddings along the 2 principal dimensions of correlation between .... TODO of an LSTM (left, $c=1$) and transformer (right, $c=3$) model.
%Color correspond the 5-means clustering of .... TODO}
%\label{fig:comm-pca}
%\end{figure*}




\section{Related work}\label{sec:related-work}

We have presented results using conditional neural language models
to model variation between speech communities.
The architecture of these models concatenates a vector representation
of the conditioned variable to the input of the sequence model.
This approach has been applied in various conditioned text generation domains such as 
image captioning \citep{Vinyals2015}, machine translation \citep{Kalchbrenner2013},
but it has not, to our knowledge, been used extensively to study linguistic variation.

There are, however, related applications of conditional neural language models.
\citet{Lau2017a} presents a neural language model that jointly learns to predict
words on the sentence-level and represent topics on the document level.
The topic representation is then fed back into the language model, 
improving its performance on next word prediction.
This is similar to how our model experiences improved performance
by learning community representations. 
Unlike our model, topics are inferred in an unsupervised way, 
raising the question of whether communities could be identified from 
unlabeled data as well.

Along these lines, \citet{OConnor2010} uses a Bayesian generative
model to infer communities from variation in text data.  In contrast
to our work, this model treats words as independent events, ignoring
the structure (and variation) in the construction of sequences.  It
does further suggest, however, that community-level variation can be
modeled in an unsupervised way.

\section{Discussion and Conclusion}\label{sec:discussion-conclusion}


% \jp{I propose that most of the following analysis should be spread between the relevant sections, and only have a tiny summary here + future work.}

% To sum up our findings, 
% we observed that various communities exhibit widely varying levels of
% linguistic complexity \ref{hyp:varying-complexity}. For the best model architecture,
% perplexity per-word ranges from less than \num{4} to more than \num{93} 
% (but with only 25\% of communities under 63).

% With indiscernibility, values are distributed evenly over the whole possible
% range (1 to 510). We find that CCLMs can be used to recognise certain
% communities based on latent Bayesian classification
% \ref{hyp:LMCC-works}, but not all.  In general, we observe that the least
% discernible communities all have a very conversational, general-interest
% premise. We cannot rule out that these more generic
% communities exhibit distinctive linguistic patterns, but none of our
% CCLMs are able to discern them. In contrast, the distinctive patterns
% of narrow-topic communities are just as recognizable by all our LSTM
% and Transformer models.

% We find a mild correlation (\cref{tab:ind-corrs}) between the
% perplexity of the CCLM and the indiscernibility of the corresponding
% community \ref{hyp:rich-harder-to-identify}, with factors ranging from
% 0.06 to 0.23. However, The trend is more clear when considering the
% correlation with information gain and indiscernibility (correlation
% from 0.39 to 0.46). So this means that, when trying to identify a
% community, it is not its absolute perplexity that matters, but rather
% the difference in perplexity with the baseline.

We find that the layer at which the community embedding is taken into
account ($l_c$) has a weak effect on the perplexity of the CCLM
\ref{hyp:layer-effect}
For LSTM models, the perplexity per word, averaged over
messages from all communities, was between \num{66.01} and \num{66.35}
(with \num{68.74} for the unconditioned model).
For transformer models, it varies a bit more, between \num{75.66} and \num{83.53},
but this seems to be mainly due to the poor performance of the models where
the community embedding is inserted between transformer layers
($l_c=2$ and $3$ both test above the unconditioned transformer's 
average perplexity of \num{79.13}).
This could have something to do with the community embedding and linear layer
interrupting the delicate layer normalisation between transformer encoder layers.

The pattern of information gain by community is similar across
architectures.  Communities that benefit most from the conditioned
model behave that way for both the LSTM and transformer.  However,
there are some differences.  For example, many of the communities with the biggest 
difference in information gain between the $l_c=0$ and $l_c=3$ LSTMs 
seem to be organised around trading collectables or organising virtual meetups
(\emph{friendsafari},
\emph{Fireteams},
\emph{RandomActsOfGaming},
\emph{randomactsofcsgo},
\emph{Pokemongiveaway},
\emph{ACTrade},
\emph{SVExchange})
Why was the model where the community information was incorporated before
the LSTM able to take better advantage of it for these communities in particular? 
It would be interesting to investigate these differences in future work, since it
could reveal differences in the kind of linguistic variation the
different model architectures capture.

The representation of communities built by CCLM exhibits a clear
correlation with user co-occurrence patterns
\ref{hyp:extra-linguistic-correlation}. A natural
question that may arise is how much this correlation can be explained by
user-specific linguistic patterns. We plan to control for this variable
in future work, for example by building author-conditioned language
models and testing if community vector embeddings can be determined by
the set of the vector embeddings of its members.

\bibliography{paper}
\bibliographystyle{acl_natbib}

\end{document}

% LocalWords:  CCLMs Gumperz NLP Hovy Ciot Bamman DelTredici CFB SMU
% LocalWords:  cancelled pre emptively CCLM Reddit subreddits dataset
% LocalWords:  preprocessed tokenized SpaCy tokenizer Honnibal LSTM
% LocalWords:  Hochreiter Vaswani softmax un AdamW Loshchilov IG Ppl
% LocalWords:  LM videogame embeddings PCA Kumar lstm LMCC ij th Lau
% LocalWords:  indiscernibility Vinyals Kalchbrenner OConnor
% LocalWords:  monotonicity subreddit Frobenius
