%

% File acl2020.tex
%
%% Based on the style files for ACL 2020, which were
%% Based on the style files for ACL 2018, NAACL 2018/19, which were
%% Based on the style files for ACL-2015, with some improvements
%%  taken from the NAACL-2016 style
%% Based on the style files for ACL-2014, which were, in turn,
%% based on ACL-2013, ACL-2012, ACL-2011, ACL-2010, ACL-IJCNLP-2009,
%% EACL-2009, IJCNLP-2008...
%% Based on the style files for EACL 2006 by 
%%e.agirre@ehu.es or Sergi.Balari@uab.es
%% and that of ACL 08 by Joakim Nivre and Noah Smith

\documentclass[11pt,a4paper]{article}
\usepackage[hyperref]{acl2020}
\usepackage[T1]{fontenc}
\usepackage{times}
\usepackage{latexsym}
\usepackage{amsmath}
\usepackage{amssymb}
\usepackage{siunitx}
\usepackage{booktabs}
\usepackage{pgfplots}
\usepackage{todonotes}
\usepackage{enumitem}
\usepackage{cleveref}
\usepackage{array}
\usepackage{makecell}
\usepackage{multirow}

\usepgfplotslibrary{groupplots}
\pgfplotsset{compat=1.16}

\renewcommand{\UrlFont}{\ttfamily\small}

\newcommand\jp[1]{\todo[backgroundcolor=blue!10]{JP: #1}}
\newcommand\bn[1]{\todo[backgroundcolor=green!10]{BN: #1}}

% This is not strictly necessary, and may be commented out,
% but it will improve the layout of the manuscript,
% and will typically save some space.
\usepackage{microtype}

%\aclfinalcopy % Uncomment this line for the final submission
%\def\aclpaperid{***} %  Enter the acl Paper ID here

%\setlength\titlebox{5cm}
% You can expand the titlebox if you need extra space
% to show all the authors. Please do not make the titlebox
% smaller than 5cm (the original size); we will check this
% in the camera-ready version and ask you to change it back.

\newcommand\Ppl{\mathsf{Ppl}}
\newcommand\IG{\mathsf{IG}}
\newcommand\Ind{\mathsf{Ind}}
\DeclareMathOperator*{\avg}{average}
\newcommand\BibTeX{B\textsc{ib}\TeX}

\newlist{hypotheses}{enumerate}{3}
\setlist[hypotheses,1]{parsep=0pt,itemsep=1pt,font=\bfseries,label=(H\arabic*)}


\newcommand{\commpcaplot}[1]{%
  \addplot[
  scatter, only marks,
  font=\tiny,
  scatter src=explicit symbolic,
  scatter/classes={
    general-interest={mark=*, blue},
    videogames={mark=*, red},
    technology={mark=*, green},
    sports={mark=*, cyan},
    female-focused={mark=*, magenta},
    other={mark=o, black}%
  },
  nodes near coords*={\community},
  visualization depends on={value \thisrow{community} \as \community}
  ] table [x=pca1, y=pca2, meta=category] {#1};
}

\newcommand{\commpcagroupplot}[2]{%
  \begin{groupplot}[
    group style={group size=2 by 1}, 
    yticklabels={,,}, xticklabels={,,},
    legend style={anchor=north, legend columns=1, font=\tiny},
    every node near coord/.append style={font=\sffamily\tiny}%
    ]
    \nextgroupplot
    \commpcaplot{#1}
    \nextgroupplot
    \commpcaplot{#2}
    \legend{General interest,Videogames,Technology,Sports,Female-focused,Other}
  \end{groupplot}
}

\newcommand{\confusionplot}[1]{%
  \addplot[
  matrix plot*,
  shader=faceted,
  mesh/cols=46,
  point meta=explicit,
  ] table [x=actual_comm, y=pred_comm, meta=#1] {floats/confusion.csv};
}

\newcommand{\confusiongroupplot}[2]{%
  \begin{groupplot}[
    group style={group size=2 by 1}, 
    enlargelimits=false,
    axis line style={draw=none},
    %axis equal,
    xmin=-1, xmax=46,
    ymin=-1, ymax=46,
    width=8cm,
    xticklabels={Advice,AskWomen,BabyBumps,CFB,Drugs,EDH,EarthPorn,Fantasy,GameDeals,GlobalOffensive,Jokes,Kappa,KerbalSpaceProgram,KotakuInAction,LifeProTips,MLS,MMA,MaddenUltimateTeam,TwoXChromosomes,Warframe,airsoft,bodybuilding,breakingmom,cars,cringe,eu4,exjw,explainlikeimfive,femalefashionadvice,food,heroesofthestorm,jailbreak,justneckbeardthings,oculus,pcmasterrace,photography,reddevils,relationships,rupaulsdragrace,stopdrinking,streetwear,techsupport,todayilearned,toronto,videos,xxfitness},
    yticklabels=none,
    xticklabel style={font=\sffamily\tiny, rotate=90},
    xtick=data,
    legend style={font=\sffamily\small},
    ]
    \nextgroupplot
    \confusionplot{#1}
    \nextgroupplot[colorbar right]
    \confusionplot{#2}
  \end{groupplot}
}

\newcommand{\pairwisecommsim}[1]{%
  \begin{tikzpicture}
    \begin{axis}[]
      \addplot[
        scatter, only marks,
        font=\tiny,
        scatter src=explicit symbolic,
        scatter/classes={
          general-interest={mark=*, blue},
          videogames={mark=*, red},
          technology={mark=*, green},
          sports={mark=*, cyan},
          female-focused={mark=*, magenta},
          other={mark=o, black},
          different={mark=x}
        },
        ] table [x=snap, y=#1, meta=meta] {floats/comm_sim.csv};
    \end{axis}
  \end{tikzpicture}
}

\newcommand{\pplscatter}[2]{%
  \begin{tikzpicture}
  \begin{axis}[]
  \addplot[
    only marks, black,
  ] table [x=#1, y=#2] {floats/cclm_lmcc_ppl.csv};
  \end{axis}
  \end{tikzpicture}
}





\title{Community-Conditioned Language Models}

\author{Anon.}

\date{}

\begin{document}
\maketitle
\begin{abstract}
  Using Community-Conditioned Language Models (CCLMs), we investigate
  the following questions: 1. Can the level of linguistic complexity
  exhibited by different communities is captured. 2. Are CCLMs
  well-tuned to various communities both by intrinsic (can we build a
  community classifier based on CCLMs) and extrinsic methods (are
  CCLMs representations correlated with other network-based
  representations of communities).
\end{abstract}
\jp{Not so good abstract}

\section{Introduction}
\jp{I am seeing computer modern rather than times when building the
  paper!?}  Linguistic communication requires that speakers share
knowledge of certain linguistic conventions, such as syntactic
structure, word meanings, and accepted patterns of interaction.
Speakers assume that these conventions are \emph{common ground} among
their interlocutors, based on joint membership in a community
\cite{Stalnaker2002, Clark1996}.  Such a community may be of any size,
from the very small, like members of a particular friend group, to the
very large, like speakers of English.  Because it is the source of
linguistic convention, the \emph{speech community} is an important
concept in the study of linguistic variation \citep{Gumperz1972}.

Not only is linguistic variation an object of inquiry in its own
right, but it is also important to consider it in the development of
robust, equitable language technology.  Previous work has shown that
by taking demographic information into account, models can achieve
better performance on various standard NLP tasks
\citep{Hovy2015,Yang2017}, but little is known about how variation at
the level of speech community affects performance.
%
There is a fair amount of work that investigates linguistic variation
using computational methods.  However, most of it considers variation
across macro-social categories, such as gender
\citep{Burger2011,Ciot2013,Bamman2014}, age \citep{Nguyen2013}, and
geographic location \citep{Eisenstein2010,Bamman2014a}.
\citet{DelTredici2017} provide an exception, by using the modified
skip-gram model of \citet{Bamman2014a} to show that lexical semantic
variation occurs across different communities organized around the
same topic.

In the present work, we further investigate linguistic variation
across online communities. In contrast to previous work, we
investigate variation across speech communities (rather than
macro-social categories), and use context-sensitive sequence models
capable of capturing variation beyond the lexical level.
%
We do so by investigate two questions.  First, we are interested in
variation in the level of linguistic complexity exhibited by different
communities.
\jp{
  Some examples (perhaps include if there is space):
  
  (CFB)
  It was against SMU in the 47-48 Cotton bowl. They had cancelled a
  game the year before against Miami for the same reason. There's
  debate on if that's really the origin for the phrase but the story
  itself is true.

  (Videos)
  Reminds me of Y2K computer bug. A lot of work went into
  pre-emptively fixing it, and a lot of bugs were found and fixed,
  some of which would have been catastrophic. Unfortunately, because
  the work was successful, a lot of people now describe it as a waste
  of time/money and a big deal made out of nothing.  }
%
Indeed, it is folklore that certain communities exhibit more
linguistic complexity than others.  Here we want to quantify this
phenomenon.  More precisely, we investigate if the degree of
complexity can be captured in a computational model.  For this purpose
define various Community Conditioned Language Models (CCLM for short),
which can be modeled as an estimator $M$ of the degree to which a
given message $m$ is acceptable in a given community $c$:
\(P_M(m \mid c)\).

Second, we want to find out how, and how much, CCLMs are attuned to
the specificity of communities. We test whether 1. the linguistic
models are correlated with a model of communities based only on
co-occurrence of users between communities and 2. if latent community
classifiers based on the CCLM can accurately identify that a given
message comes from a given community. Additionally, one may
intuitively assume that linguistically poor communities are easier to
recognize: the less linguistic diversity, the more specific the
language becomes and the easier it is to identify.

To sum up, we test the following hypotheses:
\begin{hypotheses}
\item \label{hyp:varying-complexity} Different communities have different levels of linguistic
  complexity
\item \label{hyp:layer-effect} The layer at which the community embedding is taken into account
  in a CCLM influences its perplexity.
\item \label{hyp:LMCC-works} CCLM can be used to recognize communities based on
  latent Bayesian classification.
% \item \label{hyp:LMCC-works} The layer at which the community embedding is taken into account
%   influences CCLM and its associated latent classifier.
  % This is more of a justification for studying the latent classifier than a real "user-facing" hypothesis.
\item \label{hyp:extra-linguistic-correlation} The CCLM representation of communities are correlated with
  co-occurrence of users between communities.
% \item Community stability is correlated with language complexity
\item \label{hyp:rich-harder-to-identify} Linguistically poor communities are easier to identify than
  linguistically rich communities.
\end{hypotheses}

\jp{Perhaps: Outline of the paper.}

\section{Experimental setup}

\subsection{Data sets}

We investigate linguistic variation across various communities present
on the social media website Reddit.\footnote{Comments were obtained
  from the archive at \url{https://pushshift.io/}.
  \cite{Baumgartner2020}.}
%
Reddit is divided into forums called \textit{subreddits}, 
which are typically organized around a topic of interest. 
Users create \textit{posts}, which consist of a link, image, 
or text, along with a \emph{comment} section. 
Comments are threaded: a comment can be made directly on a post,
or appear as a reply to another comment.
%
Hereafter we refer to such comments as ``messages'', to match our
convention in mathematical formulas: the letter $c$ stands for a
community, and $m$ stands for a message.

Our dataset includes messages from \num{46} subreddits, 
randomly selected from the set of all forums 
with at least \num{15000} messages per month for each month
in the three years between 2015 and 2017. 
We initially selected \num{50} subreddits, 
but excluded four of them from further analysis: 
two because they were primarily non-English and two with particularly short average message lengths.\\footnote{
The macro-average message length was \num{39.3} ($\sigma=15.7$) and the two removed communities had average comment lengths of \num{10.7} and \num{18.1}.}
% According to your footnote the removed communities are barely outliers. I wouldn't have removed them, but it is too late to do anything about that.
Each community corpus consist of \num{50000} randomly selected messages from the year 2015.

Messages were preprocessed as follows: 
We excluded the content of block quotes, code blocks, and tables.
We removed any markup (formatting) commands from the remaining text, extracting only rendered text.
We tokenized the messages using the default English model for the SpaCy tokenizer Version 2.2.3 \citep{Honnibal2017}
and lower-cased all tokens.

\subsection{Models}

We experiment with two kinds of model architecture: a simple
unidirectional LSTM \citep{Hochreiter1997} and a Transformer
\citep{Vaswani2017}.  In either case, the model is organized as a
standard $n$-layer neural sequence encoder (we set $n=3$ for all
models). As conventional, word tokens are first converted to vectors
using a trainable embedding layer. At the other end, word tokens are
predicted using a softmax projection layer. The set of word tokens is
a pre-determined vocabulary.  These encoders form the core of the
of our models, as such do not take community into account, and we call them
unconditioned models.

To condition the model (and obtain community-conditioned language
models --- CCLM), we add a \emph{community embedding} parameter to the
models.  This parameter is concatenated with the hidden layer of the
sequence encoder, for some layer $l_c \leq n$, and passed through a
linear layer which projects the resulting vector back to the original
hidden layer size.  For $l_c = n$, the output of this linear layer is
passed directly to the softmax function, just as the final hidden
layer of the sequence encoder is in other models.  For $l_c=0$, the
community embedding is concatenated with the token embedding.  For
this reason, we set the hidden size of the sequence encoder and the
size of the token to be embedding equal for all models.


\subsection{Training scheme}

We combined the 46 community corpora and split the combined corpus,
reserving 10\% for testing, 10\% for validation and the rest for training.
Splits were stratified by community so that each community had \num{5000} examples
for testing, \num{5000} for evaluation, and \num{40000} for training.

In the remainder of the paper we use $M$ to refer to messages in the
test set, and $M_j$ for the partition of the test set originating from
community $c_j$.

We use a vocabulary size of \num{40000} tokens, including a special
out-of-vocabulary token.  The vocabulary consists of the most frequent
tokens at the message level (across all communities), meaning that we
only count each token once per message.  The main reason for this
choice is to avoid over-counting otherwise rare tokens that were
repeated many times in a single message.

For the LSTM, we trained the models on a simple left-to-right language
modeling task with cross-entropy loss over the vocabulary.  Because
the transformer operates on all tokens in the sequence at once, the
language modeling was achieved by masking and incrementally un-masking
input tokens in a left-to-right fashion (so we have an
auto-regressive\jp{check} model).  We used the AdamW
\citep{Loshchilov2019} optimization algorithm, with an initial
learning rate of \num{0.001}, without any extra control on the decay
of learning rate.
%
We used a batch size of \num{256} and a maximum sequence length of
\num{64} tokens, truncating longer messages (16.8\% of messages were
longer than \num{64} tokens).  During training, a dropout rate of
$0.1$ was applied between encoder layers and after each linear layer.

All experiments use models with \num{3} sequence encoder layers,
each with hidden (and token embedding) size of \num{256}. 
The transformer models had \num{8} attention heads per layer.\footnote{
  This number of attention heads was chosen in part to give the LSTM and transformer
  models a comparable number of parameters 
  (\num{22171203} and \num{21779523}, respectively).}
The conditional models were given a community embedding with \num{16} dimensions. 
We experimented with every possible value for $l_c$ in a three-layer model ($l_c\in\{0,1,2,3\}$).

We trained the models until the validation loss stopped decreasing for
two epochs in a row, and used the weights from the epoch with the
smallest validation loss for testing.  In general, transformer models
trained for about half as long as the LSTM models (test epoch,
\cref{tab:model-results}).

\section{Results and analysis}

\subsection{Losses of CCLM}

We measure the perplexity of each example\jp{message? community?} in the test set. 
For this, we use the negative exponential of the mean cross-entropy loss 
of the language model for a given message and average over a set of messages, $M$.
\[\Ppl_M = \avg_{m\in M}(-e^{H(m)})\]
\bn{Following the logic from IG, should we change Ppl to the exponential of the average cross entropy, rather than using the average of the message-level perplexity?}
\jp{Yes}
As one would expect, 
the conditioned models have lower perplexity
than their respective unconditioned baseline models (\cref{tab:model-results}),
but this effect is different in magnitude
for models with different community embedding depths.
For the LSTM encoder, 
the best model ($l_c=1$) concatenates the community embedding after the first encoder layer,
with two subsequent LSTM layers. 
The best transformer model ($l_c=3$), on the other hand,
incorporates the community information last,
after all the sequence encoder layers.

\begin{table}
  \footnotesize
  \centering
                  
\begin{tabular}{llrrr}
\toprule
                               &       & \makecell{test \\ epoch} & $\Ppl$ & $\IG$     \\
                               & $l_c$ &         &                 &                   \\
\midrule
\multirow{5}{*}{LSTM}         & - &          21 &  51.99          &       -           \\
                              & 0 &          17 &  50.83          &    1.023          \\
                              & 1 &          34 &  49.66          &    1.047          \\
                              & 2 &          11 &  50.23          &    1.035          \\
                              & 3 &          16 &  \textbf{49.60} &    \textbf{1.048} \\
\midrule
\multirow{5}{*}{Transformer}  & - &          20 &  61.43          &        -          \\
                              & 0 &           7 &  58.71          &    1.046          \\
                              & 1 &          12 &  61.69          &    0.992          \\
                              & 2 &           7 &  78.76          &    0.780          \\
                              & 3 &          10 &  \textbf{52.28} &    \textbf{1.054} \\
\bottomrule
\end{tabular}

  \caption{
    Performance of baseline (first row for each encoder architecture) and CCLM models. 
    Perplexity and information gain are shown for the entire test set 
    (\num{5000} messages for each community).}
  \label{tab:model-results}
\end{table}

We also consider model performance stratified by community; that is,
$\Ppl_{M_j}$, where $M_j$ is the test-set messages from community
$c_j$.  We observe a great deal of variation in baseline perplexity
across communities, with $\Ppl_{M_j}$ ranging from \num{53.07} to
\num{115.29} for the LSTM model (\cref{tab:community-results}).

\subsection{Information gain of the CCLM}

Next, we consider the \emph{information gain} of each CCLM over its baseline
un-conditioned counterpart with the same sequence encoder architecture.
For a given message, information gain is defined as the difference
between the cross-entropy of the unconditioned model and the conditioned model:
\[H_{LM}(m) - H_{CCLM}(m)\]
For a set of messages, $M$, we consider the average information gain
in exponential space (as a ratio of perplexities):
\[\IG_M = \frac{e^{\avg_M(H_{LM}(m))}}{e^{\avg_M(H_{CCLM}(m))}}\]
%\jp{What we had before: average entropy gain is $H(m)-H_0(m)$, and
  %then taking the average over all messages:
  %$avg(H(m)-H_0(m)) = avg(H(m))-avg(H_0(m)) = H - H₀$. Taking the exponential
  %gives a ratio of perplexity.  Here it seems that you have commuted
  %exponential and average. It is rather weird to do average in the
  %exponential space without normalising to a probability.

  %Compare: $e^{avg(H(m))} / e^{avg(H_0(m))}$ with $avg \frac{e^{H(m)}}{e^{H₀(m)}}$.
%}

While the absolute performance of the LSTM models is better,
we find that the best transformer CCLM has higher information gain 
than the best LSTM (\cref{tab:model-results}).

The conditioned models also perform differently across different communities---%
for some communities, $\IG_{M_j}$ around \num{1}, or even slightly lower,
meaning that the CCLM has no advantage over the unconditioned baseline for that community.
For others communities it is higher, meaning that the CCLM performs better than 
the baseline for those communities (\cref{tab:community-results}).

\begin{table*}
  \centering
  \begin{tabular}{lrrrrrrrrrr}
\toprule
& \multicolumn{5}{c}{LSTM} & \multicolumn{5}{c}{Transformer} \\
    \hline
& \multirow{2}{*}{\makecell{$\Ppl$\\(baseline)}} & \multicolumn{4}{c}{$\IG$ ($l_c = $)} & \multirow{2}{*}{\makecell{$\Ppl$\\(baseline)}} & \multicolumn{4}{c}{$\IG$ ($l_c = $)} \\
    & & 0 & 1 & 2 & 3 & & 0 & 1 & 2 & 3 \\
community           &                &                     &                     &                     &                     &                       &                            &                            &                            &                            \\
\midrule
relationships       &          53.07 &                0.98 &                1.01 &                0.99 &                1.01 &                 62.19 &                       1.00 &                       0.93 &                       0.71 &                       0.99 \\
stopdrinking        &          53.27 &                1.02 &                1.04 &                1.02 &                1.03 &                 65.73 &                       1.03 &                       0.98 &                       0.76 &                       1.02 \\
Advice              &          56.22 &                0.96 &                0.99 &                0.98 &                0.99 &                 66.37 &                       0.98 &                       0.93 &                       0.69 &                       0.98 \\
BabyBumps           &          61.21 &                1.01 &                1.03 &                1.01 &                1.04 &                 71.29 &                       1.03 &                       0.96 &                       0.74 &                       1.03 \\
xxfitness           &          65.40 &                0.99 &                1.01 &                1.01 &                1.02 &                 77.25 &                       1.02 &                       0.95 &                       0.70 &                       1.01 \\
AskWomen            &          65.65 &                0.97 &                0.99 &                0.98 &                1.00 &                 77.18 &                       0.99 &                       0.93 &                       0.71 &                       0.99 \\
TwoXChromosomes     &          65.69 &                0.96 &                0.98 &                0.97 &                1.00 &                 77.46 &                       0.98 &                       0.92 &                       0.69 &                       0.97 \\
breakingmom         &          66.66 &                0.98 &                1.01 &                0.99 &                1.02 &                 77.73 &                       1.02 &                       0.95 &                       0.75 &                       1.01 \\
techsupport         &          66.74 &                1.04 &                1.06 &                1.05 &                1.05 &                 78.65 &                       1.04 &                       0.99 &                       0.74 &                       1.04 \\
femalefashionadvice &          68.42 &                1.02 &                1.04 &                1.03 &                1.05 &                 80.71 &                       1.04 &                       0.97 &                       0.73 &                       1.04 \\
GameDeals           &          69.87 &                1.03 &                1.06 &                1.04 &                1.05 &                 82.02 &                       1.05 &                       1.00 &                       0.77 &                       1.05 \\
jailbreak           &          71.95 &                1.08 &                1.10 &                1.10 &                1.09 &                 83.33 &                       1.10 &                       1.06 &                       0.84 &                       1.12 \\
EarthPorn           &          72.35 &                1.05 &                1.08 &                1.05 &                1.07 &                 86.30 &                       1.11 &                       1.03 &                       0.84 &                       1.10 \\
pcmasterrace        &          73.02 &                1.03 &                1.06 &                1.02 &                1.03 &                 85.27 &                       1.03 &                       0.99 &                       0.78 &                       1.05 \\
photography         &          73.55 &                1.01 &                1.03 &                1.02 &                1.04 &                 86.92 &                       1.03 &                       0.97 &                       0.74 &                       1.04 \\
cringe              &          74.72 &                1.01 &                1.03 &                1.02 &                1.03 &                 86.08 &                       1.03 &                       0.98 &                       0.79 &                       1.04 \\
Fantasy             &          76.72 &                1.01 &                1.03 &                1.02 &                1.04 &                 89.81 &                       1.04 &                       0.97 &                       0.75 &                       1.03 \\
EDH                 &          77.82 &                1.03 &                1.07 &                1.04 &                1.08 &                 90.46 &                       1.06 &                       0.98 &                       0.75 &                       1.09 \\
Drugs               &          78.12 &                1.01 &                1.03 &                1.03 &                1.04 &                 91.87 &                       1.03 &                       0.98 &                       0.77 &                       1.05 \\
exjw                &          78.26 &                0.99 &                1.01 &                1.01 &                1.02 &                 90.14 &                       1.00 &                       0.95 &                       0.75 &                       1.01 \\
Jokes               &          79.14 &                1.03 &                1.06 &                1.05 &                1.05 &                 90.71 &                       1.08 &                       1.04 &                       0.85 &                       1.09 \\
LifeProTips         &          79.40 &                0.96 &                0.99 &                0.98 &                1.00 &                 92.42 &                       0.99 &                       0.94 &                       0.72 &                       0.99 \\
justneckbeardthings &          80.94 &                1.02 &                1.05 &                1.04 &                1.04 &                 95.05 &                       1.05 &                       1.00 &                       0.80 &                       1.06 \\
videos              &          81.93 &                0.98 &                1.01 &                0.99 &                1.01 &                 95.32 &                       1.01 &                       0.96 &                       0.76 &                       1.02 \\
KerbalSpaceProgram  &          82.67 &                1.05 &                1.06 &                1.06 &                1.08 &                 95.33 &                       1.06 &                       1.01 &                       0.80 &                       1.07 \\
oculus              &          83.10 &                1.01 &                1.03 &                1.03 &                1.04 &                 96.75 &                       1.03 &                       0.97 &                       0.76 &                       1.04 \\
airsoft             &          83.71 &                1.04 &                1.06 &                1.05 &                1.08 &                 96.45 &                       1.06 &                       1.02 &                       0.80 &                       1.07 \\
bodybuilding        &          83.78 &                1.02 &                1.05 &                1.04 &                1.05 &                 98.98 &                       1.05 &                       1.00 &                       0.79 &                       1.06 \\
reddevils           &          84.25 &                1.06 &                1.08 &                1.07 &                1.08 &                100.35 &                       1.10 &                       1.05 &                       0.84 &                       1.11 \\
food                &          84.59 &                1.06 &                1.07 &                1.06 &                1.06 &                 97.96 &                       1.08 &                       1.02 &                       0.84 &                       1.09 \\
toronto             &          86.71 &                0.98 &                1.00 &                1.00 &                1.01 &                100.61 &                       1.00 &                       0.95 &                       0.74 &                       1.01 \\
cars                &          88.26 &                1.03 &                1.05 &                1.04 &                1.06 &                104.10 &                       1.07 &                       1.01 &                       0.79 &                       1.06 \\
rupaulsdragrace     &          88.69 &                1.05 &                1.09 &                1.07 &                1.08 &                104.58 &                       1.09 &                       1.03 &                       0.85 &                       1.11 \\
todayilearned       &          91.16 &                0.97 &                1.00 &                0.98 &                1.00 &                105.80 &                       1.00 &                       0.95 &                       0.74 &                       1.00 \\
explainlikeimfive   &          92.16 &                0.96 &                0.98 &                0.98 &                1.00 &                105.09 &                       0.98 &                       0.92 &                       0.69 &                       0.97 \\
eu4                 &          92.30 &                1.07 &                1.08 &                1.08 &                1.10 &                108.44 &                       1.10 &                       1.05 &                       0.83 &                       1.11 \\
heroesofthestorm    &          92.81 &                1.04 &                1.06 &                1.04 &                1.06 &                106.49 &                       1.04 &                       0.99 &                       0.77 &                       1.05 \\
MLS                 &          93.26 &                1.06 &                1.10 &                1.07 &                1.09 &                110.06 &                       1.08 &                       1.04 &                       0.84 &                       1.10 \\
KotakuInAction      &          94.01 &                0.98 &                1.01 &                1.00 &                1.02 &                110.13 &                       1.01 &                       0.95 &                       0.74 &                       1.01 \\
GlobalOffensive     &          94.33 &                1.05 &                1.07 &                1.06 &                1.07 &                107.81 &                       1.05 &                       1.01 &                       0.80 &                       1.08 \\
MaddenUltimateTeam  &          95.66 &                1.11 &                1.13 &                1.12 &                1.10 &                115.78 &                       1.15 &                       1.10 &                       0.90 &                       1.17 \\
MMA                 &          95.99 &                1.05 &                1.08 &                1.07 &                1.09 &                114.21 &                       1.09 &                       1.03 &                       0.84 &                       1.09 \\
CFB                 &          99.17 &                1.07 &                1.10 &                1.08 &                1.09 &                110.91 &                       1.08 &                       1.04 &                       0.85 &                       1.09 \\
streetwear          &          99.64 &                1.13 &                1.15 &                1.13 &                1.12 &                122.14 &                       1.16 &                       1.10 &                       0.92 &                       1.20 \\
Warframe            &         100.37 &                1.04 &                1.07 &                1.06 &                1.08 &                115.74 &                       1.06 &                       1.01 &                       0.80 &                       1.07 \\
Kappa               &         115.29 &                1.08 &                1.11 &                1.10 &                1.10 &                136.14 &                       1.08 &                       1.05 &                       0.87 &                       1.14 \\
\bottomrule
\end{tabular}

  \caption{
    Model perplexity and information gain by community. We show the absolute perplexity 
    of the baseline models and the information gain of the CCLMs.
    Rows are sorted by the perplexity of the baseline
    LSTM model.
}
  \label{tab:community-results}
\end{table*}


As an informal observation, we note that across all the models we
tested, communities where conditioning has the least effect tend to be
more organized around more general interest topics, such as the
\emph{relationships} and \emph{advice} subreddits, where the subject
matter is relevant to a broad range of people.  Conditioning the model
on community has the most benefit for narrower special-interest
subreddits, such as those organized around a certain videogame, sports
team, or subculture.  This makes sense intuitively, since communities
with more niche topics would tend to have more specialized vocabulary
and other linguistic patterns.

\subsection{Analysis of CCLM community embeddings}

As a qualitative assessment of the community embeddings, 
we manually identified topic categories into which at least three
of the selected communities fall.
We plotted the two-component PCA projection of the embeddings,
annotated with these categories (\cref{fig:comm-pca}).
We see that there is some evidence of clustering according to
the topic categories we identified,
suggesting that the embedding space learned by the model is meaningful.

\begin{figure*}
\begin{tikzpicture}
  \begin{groupplot}[
    group style={group size=2 by 1},
    yticklabels={,,}, xticklabels={,,},
    legend style={anchor=north, legend columns=1, font=\sffamily\tiny},
  ]
    \nextgroupplot
    \commscatter{lstm-3-1_pca1}{lstm-3-1_pca2}
    \nextgroupplot
    \commscatter{transformer-3-3_pca1}{transformer-3-3_pca2}
    \legend{General interest,Videogames,Technology,Sports,Female-focused,Other}
  \end{groupplot}
\end{tikzpicture}
\caption{Community embedding PCA of the best LSTM (left, $c=1$) and transformer (right, $c=3$) models.}
\label{fig:comm-pca}
\end{figure*}

As a quantitative assessment, we compare the CCLM-learned community
embedding with the community embedding created by \citet{Kumar2018},
who generated them using using a negative-sampling optimization
algorithm and the message-community co-occurrence matrix as ground
truth, using data which spans January 2014 to April 2017.
We refer the reader
to \citet{Kumar2018}
for details, but the important point is that no linguistic
information is used to create these embeddings: they only depend on
whether a given user participates in a given
community. In sharp contrasts, CCLM community embeddings depend in no
way on which user is the author of any given message: we only use the
contents of messages, not authorship data.
%
We correlate the pairwise cosine similarity of the
CCLM embeddings with the same metric on the membership embedding,
restricted to the 46 communities chosen for our study
(\cref{fig:pairwise-comm-sim}).

\begin{figure}
  \pairwisecommsim{snap_cos_sim}{lstm-3-1_cos_sim} 
  \caption{Pairwise cosine similarity between community vectors for 
    the membership-based embedding (x-axis) and
  the LSTM CCLM embedding with $c=1$ (y-axis). 
  Pairs from the same category are colored according to \cref{fig:comm-pca}.
  We also mark pairs from different categories (\texttimes) and pairs where one or both
  communities belong to the ``Other'' category (\textbigcircle).
  }
  \label{fig:pairwise-comm-sim}
\end{figure}

\begin{table}
  \centering
  \begin{tabular}{lrrrr}
\toprule
c &    0 &    1 &    2 &    3 \\
\midrule
LSTM        & 0.40 & 0.40 & 0.47 & 0.48 \\
Transformer & 0.48 & 0.46 & 0.46 & 0.47 \\
\bottomrule
\end{tabular}

  \caption{Pearson's $r$ between the pairwise similarity of community vectors
  in the CCLM models and the membership-based embedding of \citet{Kumar2018}.
  ($p<0.001$ for all models).
  }
  \label{tab:pairwise-comm-sim}
\end{table}

\subsection{Latent language-model-based community classifiers}

The perplexity of a CCLM gives us a measure of linguistic complexity,
but it is not a measure of how linguistically \emph{specific} the
community is. However, the CCLM induces a latent classifier,
which we call a language-model-based community classifier
(LMCC). Using the Bayes theorem, we can turn a
CCLM into a LMCC: the probability that a given message $m$
belongs to a community $c_i$ is calculated as follows:
\[P(c=c_i \mid m) = P(m | c_i)\frac {P(m)} {P(c=c_i)}\]
In the above,
$P(c=c_i)$ is the frequency of community $c_i$ in the dataset, and
$P(m \mid c_i)$ (the probability of a message $m$
in the context of community $c_i$) is estimated by the CCLM. 
$P(m)$ is the absolute
probability of a message, which can be computed as
\[P(m) = \sum_i P(c=c_i) P(m\mid c=c_i ). \]
Since the test set is balanced across communities,
we have an even prior for $P(c=c_i)$, giving us
\[P(c=c_i\mid m) \propto  P(m\mid c_i).\]

We can characterize the quality the LMCC with a confusion matrix
$C$, defined by $C_{ij} = P(c=c_i \mid m)$ for a given (test set) message $m$ 
from community $c_j$.
\[C_{ij} = \avg_{m\in M_j}P(c=c_i \mid m)\]
We observe that the diagonal dominates the matrix: the LMCC generally finds the right community for a message (\cref{fig:confusion}).
%
\begin{figure*}
\begin{tikzpicture}
  \confusiongroupplot{lstm-3-1}{transformer-3-3}
\end{tikzpicture}
\caption{LMCC confusion matrices for the best LSTM (left, $c=1$) and transformer (right, $c=3$) models.}
\label{fig:confusion}
\end{figure*}
%
But we want a more quantitative measure, corresponding to the ability of the LMCC to correctly
recognize messages coming from a given community $c_j$. We could use
something like an F-score, but we prefer to use a perplexity-like metric,
which is has stronger information-theoretic grounds.
To this end, we define the \emph{linguistic indiscernibility} of community $c_j$, 
noted $\Ind_j$. 
First, note that each row of $C$ is a probability distribution,
since it is computed as the average of $P(c=c_i \mid m)$ for $m\in M_j$.
The exponential of the entropy of the $j$th row of \(C\) is therefore a perplexity score,
whose maximum is the total number of communities (46 for us).
Thus we define $\Ind_j$ by scaling this perplexity by the number of communities:
\[\Ind_j = \frac{e^{-\sum_i C_{ij} log(C_{ij})}}{|C_j|}\]
This way $Ind_j=0$ means that the LMCC is completely certain of its
choice, while $Ind_j=1$ means that equal probability is given to all
predictions.  In general $\Ind_j$ is a score of uncertainty of the
LMCC, given messages from a community $c_j$.

Indiscernibility of a given community is a robust measure: it is
almost perfectly correlated across CCLMs (such as transformers, LSTM
and various values of $l_c$).  In fact, we observe a correlation
between the indiscernibility measures by every pair of models with
Pearson's $r \geq 0.98$ and $p < 0.0001$ (see, for example,
\cref{fig:lmcc-ppl}).

\begin{figure}
  \begin{tikzpicture}
    \begin{axis}
    \commscatter{lstm-3-1_indisc}{transformer-3-3_indisc}
    \end{axis}
  \end{tikzpicture}
  \caption{%
    Correlation between indiscernibility for 
    the best LSTM ($c = 1$) and transformer ($c = 3$) models
    (Pearson's $r = 0.99$, $p < 0.0001$).
  }
  \label{fig:lmcc-ppl}
\end{figure}

However, we observe that community-wise perplexity is only weakly (if at all) 
correlated with community indiscernibility under the corresponding latent 
LMCC (see \cref{tab:ind-corrs} and, for example, \cref{fig:cclm-lmcc-ppl}).
In other words, the linguistic complexity of a given community
is a poor predictor of how distinctive its language is.

\begin{table}
  \centering
  \begin{tabular}{llrrrr}
\toprule
    & \multirow{2}{*}{$l_c$} & \multicolumn{2}{c}{$\Ppl_{M_j}$} & \multicolumn{2}{c}{$\IG_{M_j}$} \\
                              &       &         $p$ &            $r$ &           $p$ &          $r$ \\
\midrule
\multirow{4}{*}{LSTM}         & 0 &            - &              - &             - &             - \\
                              & 1 &         0.19 &          0.000 &         -0.40 &         0.000 \\
                              & 2 &         0.19 &          0.000 &         -0.43 &         0.000 \\
                              & 3 &         0.06 &          0.218 &         -0.45 &         0.000 \\
\midrule
\multirow{4}{*}{Transformer}  & 0 &         0.19 &          0.000 &         -0.41 &         0.000 \\
                              & 1 &         0.23 &          0.000 &         -0.43 &         0.000 \\
                              & 2 &         0.16 &          0.000 &         -0.39 &         0.000 \\
                              & 3 &         0.12 &          0.010 &         -0.40 &         0.000 \\
\bottomrule
\end{tabular}


  \caption{ Pearson's $r$ correlation coefficient between community
    indiscernibility ($\Ind_j$) and two different predictors: CCLM
    perplexity on messages from community $c_j$ ($\Ppl_{M_j}$), and
    CCLM information gain on messages from $c_j$ ($\IG_{M_j}$).  }
  \label{tab:ind-corrs}
\end{table}


\begin{figure*}
  \begin{tikzpicture}
  \begin{groupplot}[
    group style={group size=2 by 1},
    legend style={anchor=north, legend columns=1, font=\tiny},
    every node near coord/.append style={font=\sffamily\tiny}%
  ]
    \nextgroupplot
    \commscatter[]{lstm-3-1_lm_ppl}{lstm-3-1_indisc}
    \nextgroupplot
    \commscatter[]{lstm-3-1_info_gain}{lstm-3-1_indisc}
    \legend{General interest,Videogames,Technology,Sports,Female-focused,Other}
  \end{groupplot}
  \end{tikzpicture}
  \caption{%
    Mean community indiscernibility (y-axis), 
    in relationship to linguistic complexity (left) and 
    CCLM information gain (right). Shown here for the LSTM
    model with $l_c=1$. See \cref{tab:ind-corrs} for 
    Pearson's r correlations for all models.
  }
  \label{fig:cclm-lmcc-ppl}
\end{figure*}


% We could also check if communities are well-separated in the SNAP community encodings

On the other hand, community indiscernibility is correlated with
the information gain of the CCLM
on messages from that community's test set
(see \cref{tab:ind-corrs} and, for example, \cref{fig:cclm-lmcc-ppl}).
Communities for which the language model benefits most from 
the conditioned model are generally easier for the LMCC to classify.

% Finally, we investigate the correlation between the stability of a
% community and the perplexity of the LMCC for it. Stability is defined as TODO.

% - compute pearson correlation coefficient

% We see a positive correlation: this suggests that the more stable a
% community is, the more difficult it is to identify message as coming
% from it. What this suggests is that stable communities tend to use a
% varied and standard, subset of English, while unstable communities
% resort to more formulaic language. This may suggests that to 'fit' in an
% unstable community, one must use more obvious linguistic cues than in
% a stable community. In fact, it is reasonable to think that a new
% member will ostensibly use community-specifc language, but once
% established in a community, a member would tend not to do so.

\section{Related work}

In this work, we present results using a conditioned neural language models
to model variation between speech communities.
The architecture of these models concatenates a vector representation
of the conditioned variable to the input of the sequence model.
This approach has been applied in various conditioned text generation domains such as 
image captioning \citep{Vinyals2015}, machine translation \citep{Kalchbrenner2013},
but it has not, to our knowledge, been used extensively to study linguistic variation.

There are, however, related applications of conditional neural language models.
\citet{Lau2017a} presents a neural language model that jointly learns to predict
words on the sentence-level and represent topics on the document level.
The topic representation is then fed back into the language model, 
improving its performance on next word prediction.
This is similar to how our model experiences improved performance
by learning community representations. 
Unlike our model, topics are inferred in an unsupervised way, 
raising the question of whether communities could be identified from 
unlabeled data as well.

Along these lines, \citet{OConnor2010} uses a Bayesian generative
model to infer communities from variation in text data.  In contrast
to our work, this model treats words as independent events, ignoring
the structure (and variation) in the construction of sequences.  It
does further suggest, however, that community-level variation could be
modeled in an unsupervised way.

\section{Discussion and Conclusion}

We find that various communities exhibit widely varying levels of
linguistic complexity \ref{hyp:varying-complexity}, with perplexity
per-word ranging from less than 53 to more than 104 for the best model
architecture.

With indiscernibility values evenly ranging over the whole possible
range (0 to 1), we find that CCLMs can be used to recognize certain
communities based on latent Bayesian classification
\ref{hyp:LMCC-works}, but not all.  If generic communities exhibit
distinctive linguistic patterns, none of our models is able to discern
them. In contrast, the distinctive patterns of narrow-topic
communities are just as recognizable by all our LSTM and Transformer
models.

We find no statistically significant correlation (\cref{tab:ind-corrs}) between the
perplexity of the
% \bn{I'm not sure if we can make this claim or not with the Ppl-Ppl correlations. However, given that *all* of them have positive correlation, even though the p-values are somewhat higher than .1 maybe we can?}
CCLM and the indiscernibility of the corresponding community
\ref{hyp:rich-harder-to-identify}. Because we have relatively few, and
only large communities in our test set, it is difficult to say if this
finding generalizes. Regardless, this kind of correlation would suggest that
community-specific complexity can be at least partially attributed to
using more varied patterns. This is view is supported by the
observation that, with an indiscernibility $> 0.9$, the most complex
communities appear as an even mixture of the kind of language found in
all communities, to our models.

We find that the layer at which the community embedding is taken into
account in a CCLM influences its perplexity \ref{hyp:layer-effect},
albeit weakly. For LSTM models, the perplexity per word, averaged over
all communities, varies between 76.80 and 78.64. (with 80.70 for the
unconditioned model) The perplexity per word is slightly higher for
$l_c=0$, and has very little difference between the other models, with
$l_c=1$ being the best.  For transformer models, it varies between
89.38 and 122.56. Here perplexity per-word is clearly not a monotonous
function of $l_c$: the maximum is reached at $l_c=2$. We did not
expect non-monotonicity, and leave studying this effect to future
work.

The pattern of information gain by community is similar across
architectures.  Communities that benefit most from the conditioned
model behave that way for both the LSTM and transformer.  However,
there are some differences.  For example, the subreddit
\emph{relationships} has an information gain of \num{1.018} for the
best CCLM LSTM but only \num{0.998} for the best transformer model.
Why was the LSTM able to take advantage of the community information
for this community, where the transformer was not?  It would be
interesting to investigate these differences in future work, since it
could reveal differences in the kind of linguistic variation the
different model architectures capture.

The representation of communities built by CCLM exhibits a clear
correlation with user co-occurrence patterns
\ref{hyp:extra-linguistic-correlation}. A natural
question that may arise is how much this correlation can be explain by
user-specific linguistic patterns. We plan to control for this variable
in future work, for example by building author-conditioned language
models and testing if community vector embeddings can be determined by
the set of the vector embeddings of its members.

\bibliography{paper}
\bibliographystyle{acl_natbib}

\end{document}

% LocalWords:  CCLMs Gumperz NLP Hovy Ciot Bamman DelTredici CFB SMU
% LocalWords:  cancelled pre emptively CCLM Reddit subreddits dataset
% LocalWords:  preprocessed tokenized SpaCy tokenizer Honnibal LSTM
% LocalWords:  Hochreiter Vaswani softmax un AdamW Loshchilov IG Ppl
% LocalWords:  LM videogame embeddings PCA Kumar lstm LMCC ij th Lau
% LocalWords:  indiscernibility Vinyals Kalchbrenner OConnor
% LocalWords:  monotonicity subreddit
