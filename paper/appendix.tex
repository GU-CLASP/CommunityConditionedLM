\documentclass[11pt]{article}
\usepackage[review]{emnlp2021}
\usepackage{times}
\usepackage{latexsym}
\usepackage{amsmath}
\usepackage{amssymb}
\usepackage{siunitx}
\usepackage{booktabs}
\usepackage{xcolor}
\usepackage{longtable}
\usepackage{todonotes}
\usepackage{enumitem}
\usepackage{cleveref}
\usepackage{array}
\usepackage{makecell}
\usepackage{multirow}
\usepackage{capt-of}

\usepackage{pgfplots}
\usetikzlibrary{arrows}
\usepgfplotslibrary{groupplots}
\pgfplotsset{compat=1.16}
\pgfplotsset{every tick label/.append style={font=\footnotesize}}
\usepackage{pgfplotstable}

\renewcommand{\UrlFont}{\ttfamily\small}

\newcommand\jp[1]{\todo[backgroundcolor=blue!10]{JP: #1}}
\newcommand\bn[1]{\todo[backgroundcolor=green!10]{BN: #1}}

% This is not strictly necessary, and may be commented out,
% but it will improve the layout of the manuscript,
% and will typically save some space.
\usepackage{microtype}

%\aclfinalcopy % Uncomment this line for the final submission
%\def\aclpaperid{***} %  Enter the acl Paper ID here

%\setlength\titlebox{5cm}
% You can expand the titlebox if you need extra space
% to show all the authors. Please do not make the titlebox
% smaller than 5cm (the original size); we will check this
% in the camera-ready version and ask you to change it back.

\newcommand\Ppl{\mathsf{Ppl}}
\newcommand\IG{\mathsf{IG}}
\newcommand\Ind{\mathsf{Ind}}
\DeclareMathOperator*{\avg}{average}
\newcommand\BibTeX{B\textsc{ib}\TeX}

\newlist{hypotheses}{enumerate}{3}
\setlist[hypotheses,1]{parsep=0pt,itemsep=1pt,font=\bfseries,label=(H\arabic*)}

\title{Community-Conditioned Language Models\\ \emph{Supplementary material}}

\author{Anon.}

\date{}

\begin{document}
\maketitle

\appendix
%\bibliography{paper}
%\bibliographystyle{acl_natbib}

\section{Projection of aligned embeddings}
\newcommand{\PCAAligned}[0]{
  \begin{tikzpicture}
    \begin{groupplot}[%
      group style={
        group size=2 by 1,
        horizontal sep=0pt,
      },
      xmin=-1, xmax=1,
      ymin=-1, ymax=1,
      xtick={-0.5,0.5}, ytick={-0.5,0.5},
      axis x line=middle,
      axis y line=middle,
      axis line style={(-)},
      axis on top=true,
      width=0.55\textwidth,
      height=0.55\textwidth,
      scatter/classes={
        0={mark=x, blue},
        1={mark=+, orange},
        2={mark=o, green},
        3={mark=triangle, gray},
        4={mark=star, magenta},
        5={mark=diamond, red}
      },
      ]
      \nextgroupplot[
        %xlabel=$U_0$, ylabel=$U_1$,
        legend cell align=left,
        legend style={font=\tiny},
        legend entries = {
          offmychest confession relationships relationship\_advice SuicideWatch, 
          TrueReddit China europe nyc socialism,
          frugalmalefashion running MechanicAdvice personalfinance Flipping,
          darksouls skyrimmods dragonage Xcom MonsterHunter,
          amiugly Rateme ladybonersgw SchoolIdolFestival GoneWildPlus,
          facepalm CrappyDesign mildlyinfuriating cringepics oddlysatisfying,
        },
        %legend pos=outer south east
        legend to name=pcalegend
      ]
      \addplot[%
        scatter, only marks,
        mark options={thick, scale=0.75},
       scatter src=explicit symbolic,
        ] table [x={web-PCA0}, y={web-PCA1}, meta={web-cluster}] {floats/comm.csv};
      \nextgroupplot[
      %xlabel=$V_0$, ylabel=$V_0$
      ]
      \addplot[%
        scatter, only marks,
        mark options={thick, scale=0.75},
       scatter src=explicit symbolic,
        ] table [x={lstm-3-1-PCA0}, y={lstm-3-1-PCA1}, meta={web-cluster}] {floats/comm.csv};
    \end{groupplot}
  \end{tikzpicture} 
}

\noindent\begin{minipage}{\textwidth}
  \centering
  \PCAAligned\\
  \vspace{0.25cm}
  \ref{pcalegend}
\captionof{figure}{First two components of the aligned social (top) and linguistic (bottom) embeddings,
  where the lingusitic embedding is taken from the LTSM with $l_c=1$.
  Correlation between these directions is given by $\sigma_0 = 53.4$ and $\sigma_1 = 35.6$.
  Colors are assigned by k-means clustering of the social embedding. The legend shows the
  closest 5 communites to each cluster centroid.
  The legend shows the closest 5 communites to each cluster centroid. 
The cluster of each community is also available in \cref{sec:community-level-results} }
\label{fig:pca-aligned}
\end{minipage}

\onecolumn

\section{Community-level results} \label{sec:community-level-results}

The following table shows results at the community level. The baseline $\Ppl_{M_j}$ 
is computed from the unconditioned LSTM and the CCLM results ($\Ppl_{M_j}$, $\IG_{M_j}$,
and $\Ind_{M_j}$ use the LSTM with $l_c=1$). ``Social cluster'' is determined by 
k-means clustering of the social embedding.

\pgfplotstableset{
  begin table=\begin{longtable},
  end table=\end{longtable},
}
\pgfplotstabletypeset[
  columns={community,lstm-3-0-ppl,lstm-3-1-ppl,lstm-3-1-ig,lstm-3-1-indisc,web-cluster},
  %columns={lstm-3-0-ppl,lstm-3-1-ppl,lstm-3-1-ig,lstm-3-1-indisc},
  columns/community/.style={verb string type},
  columns/web-cluster/.style={verb string type},
  font=\small
]{floats/comm.csv}

\end{document}

